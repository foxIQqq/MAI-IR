\section*{Задание и постановка задачи}
\addcontentsline{toc}{section}{Задание и постановка задачи}

В данной лабораторной работе требуется подготовить полный набор инструментов для построения базовой поисковой системы на собственных структурах данных. Задача включает добычу корпуса документов, реализацию поискового робота для его наполнения, построение и проверку компонентов предобработки текста: токенизации и стемминга, сбор статистики по частотам слов и проверку закона Ципфа, построение булевого инвертированного индекса и реализацию булевого поиска как командной утилиты и веб-интерфейса. В отчёте необходимо подробно описать архитектуру решения, форматы данных, алгоритмы и принятые инженерные решения, а также привести результаты тестирования и автоматическую оценку качества поиска. Требуется обеспечить воспроизводимость запуска: указать точную последовательность команд для сборки и запуска, а также включить скрипты для генерации оценочных файлов (qrels/results). Итоговый документ должен содержать исходные тексты, скрипт сборки и инструкцию по запуску тестовой процедуры. В тексте отчёта должны быть приведены численные характеристики корпуса и результаты метрической оценки (P, NDCG, ERR) — либо по ручной разметке, либо по автоматически сгенерированным qrels.
\pagebreak
