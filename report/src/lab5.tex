\section{Закон Ципфа}

\subsection{Описание и цель}
Проверка закона Ципфа используется как инструмент контроля качества корпуса и корректности предобработки текстов. Закон утверждает обратную зависимость частоты слова от его ранга, и проверка этой закономерности на реальном корпусе помогает выявить ошибки токенизации, чрезмерную долю стоп-слов или проблемы с нормализацией. Для этого в процессе индексации собираются частоты всех стемм, после чего данные экспортируются в CSV и визуализируются в лог-лог шкале. Анализ графика даёт информацию о поведении распределения в вершине и хвосте, что важно для принятия инженерных решений на предыдущих этапах обработки. Проверка Zipf — это не самоцель, а способ быстро увидеть структурные проблемы корпуса и скорректировать пайплайн.

\subsection{Реализация экспорта}
Частоты собираются в словаре при индексации; после завершения индексации все пары «терм, частота» экспортируются в CSV-файл. Экспорт выполняется в отдельной функции, которая формирует файл с заголовком и затем списком пар, упрощая последующую обработку в Python. Сортировка по убыванию частоты выполняется либо на C++ (в собственном алгоритме сортировки), либо в Python-скрипте визуализации. Экспортируемый файл затем подаётся на вход визуализатору, который генерирует PNG с лог-лог графиком.

\subsection{Интерпретация результатов}
На корректно обработанном корпусе ожидается приближенная линейность на лог-лог графике; отклонения в топе часто объясняются стоп-словами, а искривления в хвосте — шумом и редкими токенами. Для практического анализа полезно смотреть графики по диапазонам рангов (топ-10, топ-100, топ-1000), поскольку глобальная картина может скрывать локальные аномалии. Если наблюдается сильное отклонение, рекомендуется вернуться к токенизации/стеммингу и проанализировать наиболее частотные стеммы вручную. Также полезно строить гистограммы длин токенов и распределение частот по документам.

\subsection{График}
\begin{figure}[ht]
\centering
\includegraphics[width=0.82\textwidth]{img/zipf_plot.png}
\caption{Log-Log график частоты слов против ранга (Zipf).}
\end{figure}

\subsection{Практические рекомендации}
После первичной проверки Zipf рекомендуется настроить стоп-лист и пороги частот, а также провести чистку корпуса от служебных и навигационных блоков. При итеративной доработке пайплайна повторная проверка Zipf служит индикатором улучшений или регрессий. Этот процесс является частью контроля качества данных перед проведением полноценной оценки поисковой выдачи по qrels.
\pagebreak
