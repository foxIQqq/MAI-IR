\section{Булев поиск}

\subsection{Описание}
Булевый поиск — это базовый режим, в котором запросы интерпретируются как булевы формулы над термами и возвращают множество документов, удовлетворяющих формуле. В проекте реализована командная утилита \code{search\_cli}, которая загружает индекс, читает запросы со стандартного ввода и печатает найденные документы, а также лёгкий веб-интерфейс, который оборачивает CLI и отображает результаты в браузере. Такой подход даёт простоту реализации и удобство тестирования: ядро поиска остаётся на C++ для скорости, а веб-интерфейс служит фронтендом. Булевый режим удобен для формальных тестов и для случаев, когда пользователю нужно строго выразить логическую комбинацию условий.

\subsection{Парсинг запросов и исполнение}
Запросы разбиваются по пробелам; поддерживаются операторы \code{AND} и \code{OR}; при отсутствии оператора применяется \code{AND}. Каждый терм токенизируется и стеммируется, затем по стемме извлекается posting list. Для выполнения операций используются классические алгоритмы: пересечение выполняется через два указателя на отсортированные списки, объединение — через merge. Эти алгоритмы имеют линейную сложность по сумме длин списков, что делает их эффективными при умеренных размерах posting lists. Для долгих списков можно применять оптимизации: сначала выбирать операцию с наименьшим списком и итеративно пересекать с более длинными, тем самым уменьшая объём работы.

\subsection{Интеграция с веб-интерфейсом и примеры}
Веб-интерфейс реализован на Flask; он запускает \code{search\_cli} как подпроцесс и передаёт запросы через stdin, затем парсит stdout и формирует HTML. Такой подход минимизирует дублирование логики и упрощает сопровождение. Ниже приведены реальные скриншоты: один показывает вывод CLI в терминале, второй — страницу веб-интерфейса с результатами. Скриншоты подтверждают, что веб-обёртка отображает те же документы, что и CLI, и обеспечивает удобный интерфейс для проверки и демонстрации.

\begin{figure}[ht]
\centering
\includegraphics[width=0.82\textwidth]{img/cli_output.png}
\caption{Пример вывода \code{search\_cli} в терминале.}
\end{figure}

\begin{figure}[ht]
\centering
\includegraphics[width=0.82\textwidth]{img/search_result_example.png}
\caption{Пример страницы веб-интерфейса с результатами поиска.}
\end{figure}

\FloatBarrier

\subsection{Ограничения и дальнейшие шаги}
Булевый поиск не ранжирует результаты и возвращает их в произвольном порядке (либо в порядке id). Для практических приложений требуется ранжирование по релевантности, в частности внедрение TF–IDF или BM25, чтобы полезные документы шли первыми. Также полезно добавить поддержку фразового поиска и proximity queries. В дополнение к булевой логике в проекте подготовлены скрипты для генерации qrels и results и модуль для расчёта P и NDCG, что позволит количественно оценивать качество при вводе ранжирования.

\pagebreak
