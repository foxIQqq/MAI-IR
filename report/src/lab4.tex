\section{Стемминг}

\subsection{Описание}
Стемминг уменьшает словоформы до базовой формы, что даёт преимущество в полноте поиска, объединяя формы слова в одну запись в индексе. В проекте реализована компактная версия алгоритма Портера, которая покрывает наиболее частые случаи и снабжена защитными проверками, исключающими избыточную агрегацию различных по смыслу слов. Функция стемминга детерминирована и работает линейно по длине слова, что важно при массовой обработке большого корпуса. В стеммер включены проверки measure, \texttt{contains\_vowel} и шаблонов CVC, что соответствует стандартной логике Portera. Стеммер интегрирован в пайплайн: каждый токен после токенизации передаётся в стеммер, а результат идёт в индекс. Для практики важно сохранять и оригинальную форму слова, чтобы можно было при выдаче показать читабельный фрагмент.

\subsection{Архитектура и шаги}
Алгоритм разделён на последовательные шаги: начальные простые трансформации, затем более сложные замены суффиксов и завершающие корректировки. Каждый шаг применяет набор правил, которые выполняются только при выполнении условий по мере слова. Структурно реализация оформлена как набор вспомогательных функций (\texttt{is\_consonant}, measure, \texttt{contains\_vowel}, cvc) и блоков применения правил. Такой подход упрощает тестирование каждого шага и упрощает понимание логики при отладке. Для некоторых кейсов добавлены комментарии в коде с примерами вход-выход, что облегчает поддержку и развитие.

\subsection{Диаграмма шагов}
\begin{center}
\begin{tikzpicture}[node distance=8mm, every node/.style={font=\small}]
  \node[draw, rounded corners] (s1) {Step 1a};
  \node[draw, rounded corners, right=6mm of s1] (s1b) {Step 1b};
  \node[draw, rounded corners, right=6mm of s1b] (s1c) {Step 1c};
  \node[draw, rounded corners, right=6mm of s1c] (s2) {Step 2};
  \node[draw, rounded corners, right=6mm of s2] (s3) {Step 3};
  \node[draw, rounded corners, right=6mm of s3] (s4) {Step 4};
  \node[draw, rounded corners, right=6mm of s4] (s5) {Step 5};
  \draw[->] (s1) -- (s1b);
  \draw[->] (s1b) -- (s1c);
  \draw[->] (s1c) -- (s2);
  \draw[->] (s2) -- (s3);
  \draw[->] (s3) -- (s4);
  \draw[->] (s4) -- (s5);
\end{tikzpicture}
\end{center}

\subsection{Тестирование и практические замечания}
Стеммер покрыт широким набором тестов; в них включены как стандартные пары (running \texttt{->} run), так и редкие исключения. Тесты помогают обнаружить случаи over-stemming и скорректировать правила так, чтобы минимизировать нежелательные объединения. Для многоязычности потребуется иное решение — лемматизация и POS-tagging, но для английского корпуса Porter's stemmer даёт хороший компромисс между сложностью и качеством. Рекомендуется анализировать частые стеммы и вручную проверять пары «стем \texttt{->} слова», чтобы понять, нет ли систематических ошибок.
\pagebreak
